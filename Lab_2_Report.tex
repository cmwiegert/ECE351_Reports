
% Default to the notebook output style

    


% Inherit from the specified cell style.




    
\documentclass[11pt]{article}

    
    
    \usepackage[T1]{fontenc}
    % Nicer default font (+ math font) than Computer Modern for most use cases
    \usepackage{mathpazo}

    % Basic figure setup, for now with no caption control since it's done
    % automatically by Pandoc (which extracts ![](path) syntax from Markdown).
    \usepackage{graphicx}
    % We will generate all images so they have a width \maxwidth. This means
    % that they will get their normal width if they fit onto the page, but
    % are scaled down if they would overflow the margins.
    \makeatletter
    \def\maxwidth{\ifdim\Gin@nat@width>\linewidth\linewidth
    \else\Gin@nat@width\fi}
    \makeatother
    \let\Oldincludegraphics\includegraphics
    % Set max figure width to be 80% of text width, for now hardcoded.
    \renewcommand{\includegraphics}[1]{\Oldincludegraphics[width=.8\maxwidth]{#1}}
    % Ensure that by default, figures have no caption (until we provide a
    % proper Figure object with a Caption API and a way to capture that
    % in the conversion process - todo).
    \usepackage{caption}
    \DeclareCaptionLabelFormat{nolabel}{}
    \captionsetup{labelformat=nolabel}

    \usepackage{adjustbox} % Used to constrain images to a maximum size 
    \usepackage{xcolor} % Allow colors to be defined
    \usepackage{enumerate} % Needed for markdown enumerations to work
    \usepackage{geometry} % Used to adjust the document margins
    \usepackage{amsmath} % Equations
    \usepackage{amssymb} % Equations
    \usepackage{textcomp} % defines textquotesingle
    % Hack from http://tex.stackexchange.com/a/47451/13684:
    \AtBeginDocument{%
        \def\PYZsq{\textquotesingle}% Upright quotes in Pygmentized code
    }
    \usepackage{upquote} % Upright quotes for verbatim code
    \usepackage{eurosym} % defines \euro
    \usepackage[mathletters]{ucs} % Extended unicode (utf-8) support
    \usepackage[utf8x]{inputenc} % Allow utf-8 characters in the tex document
    \usepackage{fancyvrb} % verbatim replacement that allows latex
    \usepackage{grffile} % extends the file name processing of package graphics 
                         % to support a larger range 
    % The hyperref package gives us a pdf with properly built
    % internal navigation ('pdf bookmarks' for the table of contents,
    % internal cross-reference links, web links for URLs, etc.)
    \usepackage{hyperref}
    \usepackage{longtable} % longtable support required by pandoc >1.10
    \usepackage{booktabs}  % table support for pandoc > 1.12.2
    \usepackage[inline]{enumitem} % IRkernel/repr support (it uses the enumerate* environment)
    \usepackage[normalem]{ulem} % ulem is needed to support strikethroughs (\sout)
                                % normalem makes italics be italics, not underlines
    \usepackage{mathrsfs}
    

    
    
    % Colors for the hyperref package
    \definecolor{urlcolor}{rgb}{0,.145,.698}
    \definecolor{linkcolor}{rgb}{.71,0.21,0.01}
    \definecolor{citecolor}{rgb}{.12,.54,.11}

    % ANSI colors
    \definecolor{ansi-black}{HTML}{3E424D}
    \definecolor{ansi-black-intense}{HTML}{282C36}
    \definecolor{ansi-red}{HTML}{E75C58}
    \definecolor{ansi-red-intense}{HTML}{B22B31}
    \definecolor{ansi-green}{HTML}{00A250}
    \definecolor{ansi-green-intense}{HTML}{007427}
    \definecolor{ansi-yellow}{HTML}{DDB62B}
    \definecolor{ansi-yellow-intense}{HTML}{B27D12}
    \definecolor{ansi-blue}{HTML}{208FFB}
    \definecolor{ansi-blue-intense}{HTML}{0065CA}
    \definecolor{ansi-magenta}{HTML}{D160C4}
    \definecolor{ansi-magenta-intense}{HTML}{A03196}
    \definecolor{ansi-cyan}{HTML}{60C6C8}
    \definecolor{ansi-cyan-intense}{HTML}{258F8F}
    \definecolor{ansi-white}{HTML}{C5C1B4}
    \definecolor{ansi-white-intense}{HTML}{A1A6B2}
    \definecolor{ansi-default-inverse-fg}{HTML}{FFFFFF}
    \definecolor{ansi-default-inverse-bg}{HTML}{000000}

    % commands and environments needed by pandoc snippets
    % extracted from the output of `pandoc -s`
    \providecommand{\tightlist}{%
      \setlength{\itemsep}{0pt}\setlength{\parskip}{0pt}}
    \DefineVerbatimEnvironment{Highlighting}{Verbatim}{commandchars=\\\{\}}
    % Add ',fontsize=\small' for more characters per line
    \newenvironment{Shaded}{}{}
    \newcommand{\KeywordTok}[1]{\textcolor[rgb]{0.00,0.44,0.13}{\textbf{{#1}}}}
    \newcommand{\DataTypeTok}[1]{\textcolor[rgb]{0.56,0.13,0.00}{{#1}}}
    \newcommand{\DecValTok}[1]{\textcolor[rgb]{0.25,0.63,0.44}{{#1}}}
    \newcommand{\BaseNTok}[1]{\textcolor[rgb]{0.25,0.63,0.44}{{#1}}}
    \newcommand{\FloatTok}[1]{\textcolor[rgb]{0.25,0.63,0.44}{{#1}}}
    \newcommand{\CharTok}[1]{\textcolor[rgb]{0.25,0.44,0.63}{{#1}}}
    \newcommand{\StringTok}[1]{\textcolor[rgb]{0.25,0.44,0.63}{{#1}}}
    \newcommand{\CommentTok}[1]{\textcolor[rgb]{0.38,0.63,0.69}{\textit{{#1}}}}
    \newcommand{\OtherTok}[1]{\textcolor[rgb]{0.00,0.44,0.13}{{#1}}}
    \newcommand{\AlertTok}[1]{\textcolor[rgb]{1.00,0.00,0.00}{\textbf{{#1}}}}
    \newcommand{\FunctionTok}[1]{\textcolor[rgb]{0.02,0.16,0.49}{{#1}}}
    \newcommand{\RegionMarkerTok}[1]{{#1}}
    \newcommand{\ErrorTok}[1]{\textcolor[rgb]{1.00,0.00,0.00}{\textbf{{#1}}}}
    \newcommand{\NormalTok}[1]{{#1}}
    
    % Additional commands for more recent versions of Pandoc
    \newcommand{\ConstantTok}[1]{\textcolor[rgb]{0.53,0.00,0.00}{{#1}}}
    \newcommand{\SpecialCharTok}[1]{\textcolor[rgb]{0.25,0.44,0.63}{{#1}}}
    \newcommand{\VerbatimStringTok}[1]{\textcolor[rgb]{0.25,0.44,0.63}{{#1}}}
    \newcommand{\SpecialStringTok}[1]{\textcolor[rgb]{0.73,0.40,0.53}{{#1}}}
    \newcommand{\ImportTok}[1]{{#1}}
    \newcommand{\DocumentationTok}[1]{\textcolor[rgb]{0.73,0.13,0.13}{\textit{{#1}}}}
    \newcommand{\AnnotationTok}[1]{\textcolor[rgb]{0.38,0.63,0.69}{\textbf{\textit{{#1}}}}}
    \newcommand{\CommentVarTok}[1]{\textcolor[rgb]{0.38,0.63,0.69}{\textbf{\textit{{#1}}}}}
    \newcommand{\VariableTok}[1]{\textcolor[rgb]{0.10,0.09,0.49}{{#1}}}
    \newcommand{\ControlFlowTok}[1]{\textcolor[rgb]{0.00,0.44,0.13}{\textbf{{#1}}}}
    \newcommand{\OperatorTok}[1]{\textcolor[rgb]{0.40,0.40,0.40}{{#1}}}
    \newcommand{\BuiltInTok}[1]{{#1}}
    \newcommand{\ExtensionTok}[1]{{#1}}
    \newcommand{\PreprocessorTok}[1]{\textcolor[rgb]{0.74,0.48,0.00}{{#1}}}
    \newcommand{\AttributeTok}[1]{\textcolor[rgb]{0.49,0.56,0.16}{{#1}}}
    \newcommand{\InformationTok}[1]{\textcolor[rgb]{0.38,0.63,0.69}{\textbf{\textit{{#1}}}}}
    \newcommand{\WarningTok}[1]{\textcolor[rgb]{0.38,0.63,0.69}{\textbf{\textit{{#1}}}}}
    
    
    % Define a nice break command that doesn't care if a line doesn't already
    % exist.
    \def\br{\hspace*{\fill} \\* }
    % Math Jax compatibility definitions
    \def\gt{>}
    \def\lt{<}
    \let\Oldtex\TeX
    \let\Oldlatex\LaTeX
    \renewcommand{\TeX}{\textrm{\Oldtex}}
    \renewcommand{\LaTeX}{\textrm{\Oldlatex}}
    % Document parameters
    % Document title
    \title{Lab\_2\_Report}
    
    
    
    
    

    % Pygments definitions
    
\makeatletter
\def\PY@reset{\let\PY@it=\relax \let\PY@bf=\relax%
    \let\PY@ul=\relax \let\PY@tc=\relax%
    \let\PY@bc=\relax \let\PY@ff=\relax}
\def\PY@tok#1{\csname PY@tok@#1\endcsname}
\def\PY@toks#1+{\ifx\relax#1\empty\else%
    \PY@tok{#1}\expandafter\PY@toks\fi}
\def\PY@do#1{\PY@bc{\PY@tc{\PY@ul{%
    \PY@it{\PY@bf{\PY@ff{#1}}}}}}}
\def\PY#1#2{\PY@reset\PY@toks#1+\relax+\PY@do{#2}}

\expandafter\def\csname PY@tok@w\endcsname{\def\PY@tc##1{\textcolor[rgb]{0.73,0.73,0.73}{##1}}}
\expandafter\def\csname PY@tok@c\endcsname{\let\PY@it=\textit\def\PY@tc##1{\textcolor[rgb]{0.25,0.50,0.50}{##1}}}
\expandafter\def\csname PY@tok@cp\endcsname{\def\PY@tc##1{\textcolor[rgb]{0.74,0.48,0.00}{##1}}}
\expandafter\def\csname PY@tok@k\endcsname{\let\PY@bf=\textbf\def\PY@tc##1{\textcolor[rgb]{0.00,0.50,0.00}{##1}}}
\expandafter\def\csname PY@tok@kp\endcsname{\def\PY@tc##1{\textcolor[rgb]{0.00,0.50,0.00}{##1}}}
\expandafter\def\csname PY@tok@kt\endcsname{\def\PY@tc##1{\textcolor[rgb]{0.69,0.00,0.25}{##1}}}
\expandafter\def\csname PY@tok@o\endcsname{\def\PY@tc##1{\textcolor[rgb]{0.40,0.40,0.40}{##1}}}
\expandafter\def\csname PY@tok@ow\endcsname{\let\PY@bf=\textbf\def\PY@tc##1{\textcolor[rgb]{0.67,0.13,1.00}{##1}}}
\expandafter\def\csname PY@tok@nb\endcsname{\def\PY@tc##1{\textcolor[rgb]{0.00,0.50,0.00}{##1}}}
\expandafter\def\csname PY@tok@nf\endcsname{\def\PY@tc##1{\textcolor[rgb]{0.00,0.00,1.00}{##1}}}
\expandafter\def\csname PY@tok@nc\endcsname{\let\PY@bf=\textbf\def\PY@tc##1{\textcolor[rgb]{0.00,0.00,1.00}{##1}}}
\expandafter\def\csname PY@tok@nn\endcsname{\let\PY@bf=\textbf\def\PY@tc##1{\textcolor[rgb]{0.00,0.00,1.00}{##1}}}
\expandafter\def\csname PY@tok@ne\endcsname{\let\PY@bf=\textbf\def\PY@tc##1{\textcolor[rgb]{0.82,0.25,0.23}{##1}}}
\expandafter\def\csname PY@tok@nv\endcsname{\def\PY@tc##1{\textcolor[rgb]{0.10,0.09,0.49}{##1}}}
\expandafter\def\csname PY@tok@no\endcsname{\def\PY@tc##1{\textcolor[rgb]{0.53,0.00,0.00}{##1}}}
\expandafter\def\csname PY@tok@nl\endcsname{\def\PY@tc##1{\textcolor[rgb]{0.63,0.63,0.00}{##1}}}
\expandafter\def\csname PY@tok@ni\endcsname{\let\PY@bf=\textbf\def\PY@tc##1{\textcolor[rgb]{0.60,0.60,0.60}{##1}}}
\expandafter\def\csname PY@tok@na\endcsname{\def\PY@tc##1{\textcolor[rgb]{0.49,0.56,0.16}{##1}}}
\expandafter\def\csname PY@tok@nt\endcsname{\let\PY@bf=\textbf\def\PY@tc##1{\textcolor[rgb]{0.00,0.50,0.00}{##1}}}
\expandafter\def\csname PY@tok@nd\endcsname{\def\PY@tc##1{\textcolor[rgb]{0.67,0.13,1.00}{##1}}}
\expandafter\def\csname PY@tok@s\endcsname{\def\PY@tc##1{\textcolor[rgb]{0.73,0.13,0.13}{##1}}}
\expandafter\def\csname PY@tok@sd\endcsname{\let\PY@it=\textit\def\PY@tc##1{\textcolor[rgb]{0.73,0.13,0.13}{##1}}}
\expandafter\def\csname PY@tok@si\endcsname{\let\PY@bf=\textbf\def\PY@tc##1{\textcolor[rgb]{0.73,0.40,0.53}{##1}}}
\expandafter\def\csname PY@tok@se\endcsname{\let\PY@bf=\textbf\def\PY@tc##1{\textcolor[rgb]{0.73,0.40,0.13}{##1}}}
\expandafter\def\csname PY@tok@sr\endcsname{\def\PY@tc##1{\textcolor[rgb]{0.73,0.40,0.53}{##1}}}
\expandafter\def\csname PY@tok@ss\endcsname{\def\PY@tc##1{\textcolor[rgb]{0.10,0.09,0.49}{##1}}}
\expandafter\def\csname PY@tok@sx\endcsname{\def\PY@tc##1{\textcolor[rgb]{0.00,0.50,0.00}{##1}}}
\expandafter\def\csname PY@tok@m\endcsname{\def\PY@tc##1{\textcolor[rgb]{0.40,0.40,0.40}{##1}}}
\expandafter\def\csname PY@tok@gh\endcsname{\let\PY@bf=\textbf\def\PY@tc##1{\textcolor[rgb]{0.00,0.00,0.50}{##1}}}
\expandafter\def\csname PY@tok@gu\endcsname{\let\PY@bf=\textbf\def\PY@tc##1{\textcolor[rgb]{0.50,0.00,0.50}{##1}}}
\expandafter\def\csname PY@tok@gd\endcsname{\def\PY@tc##1{\textcolor[rgb]{0.63,0.00,0.00}{##1}}}
\expandafter\def\csname PY@tok@gi\endcsname{\def\PY@tc##1{\textcolor[rgb]{0.00,0.63,0.00}{##1}}}
\expandafter\def\csname PY@tok@gr\endcsname{\def\PY@tc##1{\textcolor[rgb]{1.00,0.00,0.00}{##1}}}
\expandafter\def\csname PY@tok@ge\endcsname{\let\PY@it=\textit}
\expandafter\def\csname PY@tok@gs\endcsname{\let\PY@bf=\textbf}
\expandafter\def\csname PY@tok@gp\endcsname{\let\PY@bf=\textbf\def\PY@tc##1{\textcolor[rgb]{0.00,0.00,0.50}{##1}}}
\expandafter\def\csname PY@tok@go\endcsname{\def\PY@tc##1{\textcolor[rgb]{0.53,0.53,0.53}{##1}}}
\expandafter\def\csname PY@tok@gt\endcsname{\def\PY@tc##1{\textcolor[rgb]{0.00,0.27,0.87}{##1}}}
\expandafter\def\csname PY@tok@err\endcsname{\def\PY@bc##1{\setlength{\fboxsep}{0pt}\fcolorbox[rgb]{1.00,0.00,0.00}{1,1,1}{\strut ##1}}}
\expandafter\def\csname PY@tok@kc\endcsname{\let\PY@bf=\textbf\def\PY@tc##1{\textcolor[rgb]{0.00,0.50,0.00}{##1}}}
\expandafter\def\csname PY@tok@kd\endcsname{\let\PY@bf=\textbf\def\PY@tc##1{\textcolor[rgb]{0.00,0.50,0.00}{##1}}}
\expandafter\def\csname PY@tok@kn\endcsname{\let\PY@bf=\textbf\def\PY@tc##1{\textcolor[rgb]{0.00,0.50,0.00}{##1}}}
\expandafter\def\csname PY@tok@kr\endcsname{\let\PY@bf=\textbf\def\PY@tc##1{\textcolor[rgb]{0.00,0.50,0.00}{##1}}}
\expandafter\def\csname PY@tok@bp\endcsname{\def\PY@tc##1{\textcolor[rgb]{0.00,0.50,0.00}{##1}}}
\expandafter\def\csname PY@tok@fm\endcsname{\def\PY@tc##1{\textcolor[rgb]{0.00,0.00,1.00}{##1}}}
\expandafter\def\csname PY@tok@vc\endcsname{\def\PY@tc##1{\textcolor[rgb]{0.10,0.09,0.49}{##1}}}
\expandafter\def\csname PY@tok@vg\endcsname{\def\PY@tc##1{\textcolor[rgb]{0.10,0.09,0.49}{##1}}}
\expandafter\def\csname PY@tok@vi\endcsname{\def\PY@tc##1{\textcolor[rgb]{0.10,0.09,0.49}{##1}}}
\expandafter\def\csname PY@tok@vm\endcsname{\def\PY@tc##1{\textcolor[rgb]{0.10,0.09,0.49}{##1}}}
\expandafter\def\csname PY@tok@sa\endcsname{\def\PY@tc##1{\textcolor[rgb]{0.73,0.13,0.13}{##1}}}
\expandafter\def\csname PY@tok@sb\endcsname{\def\PY@tc##1{\textcolor[rgb]{0.73,0.13,0.13}{##1}}}
\expandafter\def\csname PY@tok@sc\endcsname{\def\PY@tc##1{\textcolor[rgb]{0.73,0.13,0.13}{##1}}}
\expandafter\def\csname PY@tok@dl\endcsname{\def\PY@tc##1{\textcolor[rgb]{0.73,0.13,0.13}{##1}}}
\expandafter\def\csname PY@tok@s2\endcsname{\def\PY@tc##1{\textcolor[rgb]{0.73,0.13,0.13}{##1}}}
\expandafter\def\csname PY@tok@sh\endcsname{\def\PY@tc##1{\textcolor[rgb]{0.73,0.13,0.13}{##1}}}
\expandafter\def\csname PY@tok@s1\endcsname{\def\PY@tc##1{\textcolor[rgb]{0.73,0.13,0.13}{##1}}}
\expandafter\def\csname PY@tok@mb\endcsname{\def\PY@tc##1{\textcolor[rgb]{0.40,0.40,0.40}{##1}}}
\expandafter\def\csname PY@tok@mf\endcsname{\def\PY@tc##1{\textcolor[rgb]{0.40,0.40,0.40}{##1}}}
\expandafter\def\csname PY@tok@mh\endcsname{\def\PY@tc##1{\textcolor[rgb]{0.40,0.40,0.40}{##1}}}
\expandafter\def\csname PY@tok@mi\endcsname{\def\PY@tc##1{\textcolor[rgb]{0.40,0.40,0.40}{##1}}}
\expandafter\def\csname PY@tok@il\endcsname{\def\PY@tc##1{\textcolor[rgb]{0.40,0.40,0.40}{##1}}}
\expandafter\def\csname PY@tok@mo\endcsname{\def\PY@tc##1{\textcolor[rgb]{0.40,0.40,0.40}{##1}}}
\expandafter\def\csname PY@tok@ch\endcsname{\let\PY@it=\textit\def\PY@tc##1{\textcolor[rgb]{0.25,0.50,0.50}{##1}}}
\expandafter\def\csname PY@tok@cm\endcsname{\let\PY@it=\textit\def\PY@tc##1{\textcolor[rgb]{0.25,0.50,0.50}{##1}}}
\expandafter\def\csname PY@tok@cpf\endcsname{\let\PY@it=\textit\def\PY@tc##1{\textcolor[rgb]{0.25,0.50,0.50}{##1}}}
\expandafter\def\csname PY@tok@c1\endcsname{\let\PY@it=\textit\def\PY@tc##1{\textcolor[rgb]{0.25,0.50,0.50}{##1}}}
\expandafter\def\csname PY@tok@cs\endcsname{\let\PY@it=\textit\def\PY@tc##1{\textcolor[rgb]{0.25,0.50,0.50}{##1}}}

\def\PYZbs{\char`\\}
\def\PYZus{\char`\_}
\def\PYZob{\char`\{}
\def\PYZcb{\char`\}}
\def\PYZca{\char`\^}
\def\PYZam{\char`\&}
\def\PYZlt{\char`\<}
\def\PYZgt{\char`\>}
\def\PYZsh{\char`\#}
\def\PYZpc{\char`\%}
\def\PYZdl{\char`\$}
\def\PYZhy{\char`\-}
\def\PYZsq{\char`\'}
\def\PYZdq{\char`\"}
\def\PYZti{\char`\~}
% for compatibility with earlier versions
\def\PYZat{@}
\def\PYZlb{[}
\def\PYZrb{]}
\makeatother


    % Exact colors from NB
    \definecolor{incolor}{rgb}{0.0, 0.0, 0.5}
    \definecolor{outcolor}{rgb}{0.545, 0.0, 0.0}



    
    % Prevent overflowing lines due to hard-to-break entities
    \sloppy 
    % Setup hyperref package
    \hypersetup{
      breaklinks=true,  % so long urls are correctly broken across lines
      colorlinks=true,
      urlcolor=urlcolor,
      linkcolor=linkcolor,
      citecolor=citecolor,
      }
    % Slightly bigger margins than the latex defaults
    
    \geometry{verbose,tmargin=1in,bmargin=1in,lmargin=1in,rmargin=1in}
    
    

    \begin{document}
    
    
    \maketitle
    
    

    <script>
  function code_toggle() {
    if (code_shown){
      $('div.input').hide('500');
      $('#toggleButton').val('Show Code')
    } else {
      $('div.input').show('500');
      $('#toggleButton').val('Hide Code')
    }
    code_shown = !code_shown
  }

  $( document ).ready(function(){
    code_shown=false;
    $('div.input').hide()
  });
</script>
<form action="javascript:code_toggle()"><input type="submit" id="toggleButton" value="Show Code"></form>
    ECE 351 Section 51

User-Defined Functions

Lab 2

\hypertarget{submitted-by}{%
\paragraph{Submitted by:}\label{submitted-by}}

\hypertarget{chris-wiegert}{%
\subsection{Chris Wiegert}\label{chris-wiegert}}

    \hypertarget{contents}{%
\section{Contents}\label{contents}}

\hypertarget{introduction.pg.1}{%
\paragraph{Introduction\ldots{}\ldots{}\ldots{}\ldots{}\ldots{}\ldots{}\ldots{}\ldots{}\ldots{}\ldots{}\ldots{}.pg.1}\label{introduction.pg.1}}

\hypertarget{equations..pg.2}{%
\paragraph{Equations\ldots{}\ldots{}\ldots{}\ldots{}\ldots{}\ldots{}\ldots{}\ldots{}\ldots{}\ldots{}\ldots{}\ldots{}..pg.2}\label{equations..pg.2}}

\hypertarget{methodologypg.2}{%
\paragraph{Methodology\ldots{}\ldots{}\ldots{}\ldots{}\ldots{}\ldots{}\ldots{}\ldots{}\ldots{}\ldots{}\ldots{}pg.2}\label{methodologypg.2}}

\hypertarget{resultsdeliverables..pg.3}{%
\paragraph{Results/Deliverables\ldots{}\ldots{}\ldots{}\ldots{}\ldots{}\ldots{}..pg.3}\label{resultsdeliverables..pg.3}}

\hypertarget{error-analysis.pg.7}{%
\paragraph{Error
Analysis\ldots{}\ldots{}\ldots{}\ldots{}\ldots{}\ldots{}\ldots{}\ldots{}\ldots{}\ldots{}.pg.7}\label{error-analysis.pg.7}}

\hypertarget{questionsconclusionspg.8}{%
\paragraph{Questions/Conclusions\ldots{}\ldots{}\ldots{}\ldots{}\ldots{}pg.8}\label{questionsconclusionspg.8}}

    \hypertarget{introduction}{%
\section{Introduction:}\label{introduction}}

User defined functions are immensely useful and a quick way to
manipulate data and equations. The modularity adds another level by
allowing the user to insert different sets of data to quickly yield
results without having to write custom code for each input set.

The purpose of these exercises is to get a handle on how to create and
implement user-defined functions. By creating generic functions the user
can pass any amount of desired data and let the computer handle all the
mathematics behind the scenes. This allows the user to create multiple
graphs and crunch lots of data very quickly without a whole bunch of
unnecessary code.

    \hypertarget{equations}{%
\section{Equations:}\label{equations}}

The Unit Step Function as seen below is the first equation used in order
to create a plot of the signal given in the lab handout. This equation
is defined below mathematically and says that if \$ t \$ time is greater
than or equal to zero, then \$ x \$ is equal to 1 and is zero when \$ t
\$ is less than zero.

\[ x =   \left\{
\begin{array}{ll}
      1 , t \geq 0 \\
      0 , t < 0 \\
\end{array} 
\right. \]

The Ramp Function as seen below is the next equation used to create the
plot of the signal. This equation says that if \$ t \$ time is greater
than or equal to zero, then \$ x \$ is equal to \$ t \$ and \$ x \$ is
zero when \$ t \$ is less than zero.

\[ x =   \left\{
\begin{array}{ll}
      x , t \geq 0 \\
      0 , t < 0 \\
\end{array} 
\right. \]

These to functions will be used together to plot the signal defined by
the equation seen below

\[ f(t)=r(t)-r(t-3)+5u(t-3)-2u(t-6)-2r(t-6) \]

    \hypertarget{methodology}{%
\section{Methodology}\label{methodology}}

The first step towards completing this lab was to get a handle on
creating user-defined functions by manipulating a given function to
produce a cosine wave. This entailed changing the ``sin'' part of the
code to ``cos'' and then removing the part of the loop that produced a
sharp peak on the graph.

The next step was to convert the given graph into a function that
defines the signal. With the equation worked out the next part was to
convert the mathematical definitions for the Step and Ramp Functions
into something the computer can work with and verify they produce the
proper results. For this, for-loops combined with if/else statements
were used to describe, in code, the behavior of these functions. Below
are examples of the code, with comments, used to produce the proper step
and ramp signal.

    \begin{Verbatim}[commandchars=\\\{\}]
{\color{incolor}In [{\color{incolor}1}]:} \PY{k}{def} \PY{n+nf}{stepfunc}\PY{p}{(}\PY{n}{w}\PY{p}{)}\PY{p}{:}                  \PY{c+c1}{\PYZsh{} Name for the user\PYZhy{}defined Step Function}
            \PY{n}{x}\PY{o}{=}\PY{n}{np}\PY{o}{.}\PY{n}{zeros}\PY{p}{(}\PY{p}{(}\PY{n+nb}{len}\PY{p}{(}\PY{n}{w}\PY{p}{)}\PY{p}{,}\PY{l+m+mi}{1}\PY{p}{)}\PY{p}{)}        \PY{c+c1}{\PYZsh{} Initializes x to an array of zeros}
            
            \PY{k}{for} \PY{n}{i} \PY{o+ow}{in} \PY{n+nb}{range}\PY{p}{(}\PY{n+nb}{len}\PY{p}{(}\PY{n}{w}\PY{p}{)}\PY{p}{)}\PY{p}{:}       \PY{c+c1}{\PYZsh{} Creates the for loop }
                \PY{k}{if} \PY{n}{w}\PY{p}{[}\PY{n}{i}\PY{p}{]} \PY{o}{\PYZgt{}}\PY{o}{=} \PY{l+m+mi}{0}\PY{p}{:}             \PY{c+c1}{\PYZsh{} Definition of the Step Function in code}
                    \PY{n}{x}\PY{p}{[}\PY{n}{i}\PY{p}{]} \PY{o}{=} \PY{l+m+mi}{1}
                \PY{k}{else}\PY{p}{:}
                    \PY{n}{x}\PY{p}{[}\PY{n}{i}\PY{p}{]} \PY{o}{=} \PY{l+m+mi}{0}
                    
            \PY{k}{return} \PY{n}{x}                      \PY{c+c1}{\PYZsh{} Returns the value found by the function}
            
        
        \PY{k}{def} \PY{n+nf}{rampfunc}\PY{p}{(}\PY{n}{w}\PY{p}{)}\PY{p}{:}                  \PY{c+c1}{\PYZsh{} Name for the user\PYZhy{}defined Ramp Function}
            \PY{n}{x}\PY{o}{=}\PY{n}{np}\PY{o}{.}\PY{n}{zeros}\PY{p}{(}\PY{p}{(}\PY{n+nb}{len}\PY{p}{(}\PY{n}{w}\PY{p}{)}\PY{p}{,}\PY{l+m+mi}{1}\PY{p}{)}\PY{p}{)}        \PY{c+c1}{\PYZsh{} Initializes x to an array of zeros}
            
            \PY{k}{for} \PY{n}{i} \PY{o+ow}{in} \PY{n+nb}{range}\PY{p}{(}\PY{n+nb}{len}\PY{p}{(}\PY{n}{w}\PY{p}{)}\PY{p}{)}\PY{p}{:}       \PY{c+c1}{\PYZsh{} Creates the for loop}
                \PY{k}{if} \PY{n}{w}\PY{p}{[}\PY{n}{i}\PY{p}{]} \PY{o}{\PYZgt{}}\PY{o}{=} \PY{l+m+mi}{0}\PY{p}{:}             \PY{c+c1}{\PYZsh{} Definition of the Ramp Function in code}
                    \PY{n}{x}\PY{p}{[}\PY{n}{i}\PY{p}{]} \PY{o}{=} \PY{n}{w}\PY{p}{[}\PY{n}{i}\PY{p}{]}
                \PY{k}{else}\PY{p}{:}
                    \PY{n}{x}\PY{p}{[}\PY{n}{i}\PY{p}{]} \PY{o}{=} \PY{l+m+mi}{0}
            \PY{k}{return} \PY{n}{x}                      \PY{c+c1}{\PYZsh{} Returns the value found by the function}
\end{Verbatim}

    With the code worked out and the functions working, the next step was to
create another function to plot the signal. For this, all that was
needed, other than the function definition, was the derived equation \$
f(t) \$ put into the functions return statement, properly scaled and
offset to produce the signal. Example of the code as follows.

    \begin{Verbatim}[commandchars=\\\{\}]
{\color{incolor}In [{\color{incolor}2}]:} \PY{k}{def} \PY{n+nf}{signalfunc}\PY{p}{(}\PY{n}{w}\PY{p}{)}\PY{p}{:}
            \PY{k}{return} \PY{p}{(}\PY{n}{rampfunc}\PY{p}{(}\PY{n}{w}\PY{p}{)} \PY{o}{\PYZhy{}} \PY{n}{rampfunc}\PY{p}{(}\PY{n}{w}\PY{o}{\PYZhy{}}\PY{l+m+mi}{3}\PY{p}{)} \PY{o}{+} \PY{l+m+mi}{5}\PY{o}{*}\PY{n}{stepfunc}\PY{p}{(}\PY{n}{w}\PY{o}{\PYZhy{}}\PY{l+m+mi}{3}\PY{p}{)} \PY{o}{\PYZhy{}} \PY{l+m+mi}{2}\PY{o}{*}\PY{n}{stepfunc}\PY{p}{(}\PY{n}{w}\PY{o}{\PYZhy{}}\PY{l+m+mi}{6}\PY{p}{)} \PY{o}{\PYZhy{}} \PY{l+m+mi}{2}\PY{o}{*}\PY{n}{rampfunc}\PY{p}{(}\PY{n}{w}\PY{o}{\PYZhy{}}\PY{l+m+mi}{6}\PY{p}{)}\PY{p}{)}
\end{Verbatim}

    From here out, to achieve the desired results as a simple matter of
manipulating the above functions' call symbol (w) to produce the desired
effect on the derived function. For example, to produce the plot of the
function, define a variable and set it equal to the name of the above
function such as y = signalfunc(w). Then, to produce a time reversal of
the function was a simple matter of calling the signalfunc() and putting
``-w'' in the parens.

The final part of the exercise was to plot the derivative of the
function but for this part the TA provided the code as it was not a
simple or straight-forward code for beginners to create. The objective
or that part became to gain an understanding of what was occuring in the
created graph.

    \hypertarget{resultsdeliverables}{%
\section{Results/Deliverables:}\label{resultsdeliverables}}

\hypertarget{part-1}{%
\subsection{Part 1:}\label{part-1}}

The deliverables for the first part consists of the code for the
user-defined function to create the cosine wave and then the plot of the
cosine wave. Both can be found below.

    \begin{Verbatim}[commandchars=\\\{\}]
{\color{incolor}In [{\color{incolor}3}]:} \PY{k+kn}{import} \PY{n+nn}{numpy} \PY{k}{as} \PY{n+nn}{np}
        \PY{k+kn}{import} \PY{n+nn}{matplotlib}\PY{n+nn}{.}\PY{n+nn}{pyplot} \PY{k}{as} \PY{n+nn}{plt}
        \PY{n}{plt}\PY{o}{.}\PY{n}{rcParams}\PY{o}{.}\PY{n}{update}\PY{p}{(}\PY{p}{\PYZob{}}\PY{l+s+s1}{\PYZsq{}}\PY{l+s+s1}{font.size}\PY{l+s+s1}{\PYZsq{}}\PY{p}{:} \PY{l+m+mi}{14}\PY{p}{\PYZcb{}}\PY{p}{)}
        
        \PY{n}{steps} \PY{o}{=} \PY{l+m+mf}{1e\PYZhy{}2}
        \PY{n}{t} \PY{o}{=} \PY{n}{np}\PY{o}{.}\PY{n}{arange}\PY{p}{(}\PY{l+m+mi}{0}\PY{p}{,}\PY{l+m+mi}{10}\PY{o}{+}\PY{n}{steps}\PY{p}{,}\PY{n}{steps}\PY{p}{)}
\end{Verbatim}

    \begin{Verbatim}[commandchars=\\\{\}]
{\color{incolor}In [{\color{incolor}4}]:} \PY{k}{def} \PY{n+nf}{func1}\PY{p}{(}\PY{n}{t}\PY{p}{)}\PY{p}{:}
            \PY{n}{y} \PY{o}{=} \PY{n}{np}\PY{o}{.}\PY{n}{zeros}\PY{p}{(}\PY{p}{(}\PY{n+nb}{len}\PY{p}{(}\PY{n}{t}\PY{p}{)}\PY{p}{,}\PY{l+m+mi}{1}\PY{p}{)}\PY{p}{)} \PY{c+c1}{\PYZsh{} initialize `y` as a numpy array (of zeros)}
        
            \PY{k}{for} \PY{n}{i} \PY{o+ow}{in} \PY{n+nb}{range}\PY{p}{(}\PY{n+nb}{len}\PY{p}{(}\PY{n}{t}\PY{p}{)}\PY{p}{)}\PY{p}{:}  \PY{c+c1}{\PYZsh{} creates the for loop to produce the cosine plot}
                \PY{n}{y}\PY{p}{[}\PY{n}{i}\PY{p}{]} \PY{o}{=} \PY{n}{np}\PY{o}{.}\PY{n}{cos}\PY{p}{(}\PY{n}{t}\PY{p}{[}\PY{n}{i}\PY{p}{]}\PY{p}{)}
            \PY{k}{return} \PY{n}{y}
\end{Verbatim}

    \begin{Verbatim}[commandchars=\\\{\}]
{\color{incolor}In [{\color{incolor}5}]:} \PY{n}{y} \PY{o}{=} \PY{n}{func1}\PY{p}{(}\PY{n}{t}\PY{p}{)} \PY{c+c1}{\PYZsh{} function call using the user\PYZhy{}defined function, shown in the above cell}
        \PY{n}{myFigSize} \PY{o}{=} \PY{p}{(}\PY{l+m+mi}{10}\PY{p}{,}\PY{l+m+mi}{8}\PY{p}{)}
        \PY{n}{plt}\PY{o}{.}\PY{n}{figure}\PY{p}{(}\PY{n}{figsize}\PY{o}{=}\PY{n}{myFigSize}\PY{p}{)}
        \PY{n}{plt}\PY{o}{.}\PY{n}{subplot}\PY{p}{(}\PY{l+m+mi}{2}\PY{p}{,}\PY{l+m+mi}{1}\PY{p}{,}\PY{l+m+mi}{1}\PY{p}{)}
        \PY{n}{plt}\PY{o}{.}\PY{n}{plot}\PY{p}{(}\PY{n}{t}\PY{p}{,}\PY{n}{y}\PY{p}{)}
        \PY{n}{plt}\PY{o}{.}\PY{n}{grid}\PY{p}{(}\PY{k+kc}{True}\PY{p}{)}
        \PY{n}{plt}\PY{o}{.}\PY{n}{ylabel}\PY{p}{(}\PY{l+s+s1}{\PYZsq{}}\PY{l+s+s1}{y(t) with Good Resolution}\PY{l+s+s1}{\PYZsq{}}\PY{p}{)}
        \PY{n}{plt}\PY{o}{.}\PY{n}{title}\PY{p}{(}\PY{l+s+s1}{\PYZsq{}}\PY{l+s+s1}{Background \PYZhy{} Illustration of for Loops and if/else Statements}\PY{l+s+s1}{\PYZsq{}}\PY{p}{)} \PY{c+c1}{\PYZsh{} `t` defined with poor resolution}
        \PY{n}{plt}\PY{o}{.}\PY{n}{show}\PY{p}{(}\PY{p}{)} \PY{c+c1}{\PYZsh{} this must be used for each figure}
\end{Verbatim}

    \begin{center}
    \adjustimage{max size={0.9\linewidth}{0.9\paperheight}}{output_13_0.png}
    \end{center}
    { \hspace*{\fill} \\}
    
    \hypertarget{part-2}{%
\subsection{Part 2:}\label{part-2}}

For part 2, the deliverables required were a properly formatted derived
function for the given plot, the code for the user-defined functions for
the Step and Ramp Functions, and the code for the final function along
with the output plot. The derived equation is seen below.

\[f(t)=r(t)-r(t-3)+5u(t-3)-2u(t-6)-2r(t-6)\]

Below is the code for the Step and Ramp functions, labeled appropriately
followed by the code for the production of the plot from the derived
equation \$ f(t) \$ and the corresponding graph.

    \begin{Verbatim}[commandchars=\\\{\}]
{\color{incolor}In [{\color{incolor}6}]:} \PY{k}{def} \PY{n+nf}{stepfunc}\PY{p}{(}\PY{n}{w}\PY{p}{)}\PY{p}{:}                  \PY{c+c1}{\PYZsh{} Name for the user\PYZhy{}defined Step Function}
            \PY{n}{x}\PY{o}{=}\PY{n}{np}\PY{o}{.}\PY{n}{zeros}\PY{p}{(}\PY{p}{(}\PY{n+nb}{len}\PY{p}{(}\PY{n}{w}\PY{p}{)}\PY{p}{,}\PY{l+m+mi}{1}\PY{p}{)}\PY{p}{)}        \PY{c+c1}{\PYZsh{} Initializes x to an array of zeros}
            
            \PY{k}{for} \PY{n}{i} \PY{o+ow}{in} \PY{n+nb}{range}\PY{p}{(}\PY{n+nb}{len}\PY{p}{(}\PY{n}{w}\PY{p}{)}\PY{p}{)}\PY{p}{:}       \PY{c+c1}{\PYZsh{} Creates the for loop }
                \PY{k}{if} \PY{n}{w}\PY{p}{[}\PY{n}{i}\PY{p}{]} \PY{o}{\PYZgt{}}\PY{o}{=} \PY{l+m+mi}{0}\PY{p}{:}             \PY{c+c1}{\PYZsh{} Definition of the Step Function in code}
                    \PY{n}{x}\PY{p}{[}\PY{n}{i}\PY{p}{]} \PY{o}{=} \PY{l+m+mi}{1}
                \PY{k}{else}\PY{p}{:}
                    \PY{n}{x}\PY{p}{[}\PY{n}{i}\PY{p}{]} \PY{o}{=} \PY{l+m+mi}{0}
                    
            \PY{k}{return} \PY{n}{x}                      \PY{c+c1}{\PYZsh{} Returns the value found by the function}
            
        
        \PY{k}{def} \PY{n+nf}{rampfunc}\PY{p}{(}\PY{n}{w}\PY{p}{)}\PY{p}{:}                  \PY{c+c1}{\PYZsh{} Name for the user\PYZhy{}defined Ramp Function}
            \PY{n}{x}\PY{o}{=}\PY{n}{np}\PY{o}{.}\PY{n}{zeros}\PY{p}{(}\PY{p}{(}\PY{n+nb}{len}\PY{p}{(}\PY{n}{w}\PY{p}{)}\PY{p}{,}\PY{l+m+mi}{1}\PY{p}{)}\PY{p}{)}        \PY{c+c1}{\PYZsh{} Initializes x to an array of zeros}
            
            \PY{k}{for} \PY{n}{i} \PY{o+ow}{in} \PY{n+nb}{range}\PY{p}{(}\PY{n+nb}{len}\PY{p}{(}\PY{n}{w}\PY{p}{)}\PY{p}{)}\PY{p}{:}       \PY{c+c1}{\PYZsh{} Creates the for loop}
                \PY{k}{if} \PY{n}{w}\PY{p}{[}\PY{n}{i}\PY{p}{]} \PY{o}{\PYZgt{}}\PY{o}{=} \PY{l+m+mi}{0}\PY{p}{:}             \PY{c+c1}{\PYZsh{} Definition of the Ramp Function in code}
                    \PY{n}{x}\PY{p}{[}\PY{n}{i}\PY{p}{]} \PY{o}{=} \PY{n}{w}\PY{p}{[}\PY{n}{i}\PY{p}{]}
                \PY{k}{else}\PY{p}{:}
                    \PY{n}{x}\PY{p}{[}\PY{n}{i}\PY{p}{]} \PY{o}{=} \PY{l+m+mi}{0}
            \PY{k}{return} \PY{n}{x}                      \PY{c+c1}{\PYZsh{} Returns the value found by the function}
        
        \PY{k}{def} \PY{n+nf}{signalfunc}\PY{p}{(}\PY{n}{w}\PY{p}{)}\PY{p}{:}                \PY{c+c1}{\PYZsh{} Creation of the function to produce the derived function graph}
            \PY{k}{return} \PY{p}{(}\PY{n}{rampfunc}\PY{p}{(}\PY{n}{w}\PY{p}{)} \PY{o}{\PYZhy{}} \PY{n}{rampfunc}\PY{p}{(}\PY{n}{w}\PY{o}{\PYZhy{}}\PY{l+m+mi}{3}\PY{p}{)} \PY{o}{+} \PY{l+m+mi}{5}\PY{o}{*}\PY{n}{stepfunc}\PY{p}{(}\PY{n}{w}\PY{o}{\PYZhy{}}\PY{l+m+mi}{3}\PY{p}{)} \PY{o}{\PYZhy{}} \PY{l+m+mi}{2}\PY{o}{*}\PY{n}{stepfunc}\PY{p}{(}\PY{n}{w}\PY{o}{\PYZhy{}}\PY{l+m+mi}{6}\PY{p}{)} \PY{o}{\PYZhy{}} \PY{l+m+mi}{2}\PY{o}{*}\PY{n}{rampfunc}\PY{p}{(}\PY{n}{w}\PY{o}{\PYZhy{}}\PY{l+m+mi}{6}\PY{p}{)}\PY{p}{)}
\end{Verbatim}

    \begin{Verbatim}[commandchars=\\\{\}]
{\color{incolor}In [{\color{incolor}7}]:} \PY{k}{def} \PY{n+nf}{signalfunc}\PY{p}{(}\PY{n}{w}\PY{p}{)}\PY{p}{:}
            \PY{k}{return} \PY{p}{(}\PY{n}{rampfunc}\PY{p}{(}\PY{n}{w}\PY{p}{)} \PY{o}{\PYZhy{}} \PY{n}{rampfunc}\PY{p}{(}\PY{n}{w}\PY{o}{\PYZhy{}}\PY{l+m+mi}{3}\PY{p}{)} \PY{o}{+} \PY{l+m+mi}{5}\PY{o}{*}\PY{n}{stepfunc}\PY{p}{(}\PY{n}{w}\PY{o}{\PYZhy{}}\PY{l+m+mi}{3}\PY{p}{)} \PY{o}{\PYZhy{}} \PY{l+m+mi}{2}\PY{o}{*}\PY{n}{stepfunc}\PY{p}{(}\PY{n}{w}\PY{o}{\PYZhy{}}\PY{l+m+mi}{6}\PY{p}{)} \PY{o}{\PYZhy{}} \PY{l+m+mi}{2}\PY{o}{*}\PY{n}{rampfunc}\PY{p}{(}\PY{n}{w}\PY{o}{\PYZhy{}}\PY{l+m+mi}{6}\PY{p}{)}\PY{p}{)}
        
        \PY{n}{steps} \PY{o}{=} \PY{l+m+mf}{1e\PYZhy{}3}
        \PY{n}{w} \PY{o}{=} \PY{n}{np}\PY{o}{.}\PY{n}{arange}\PY{p}{(}\PY{o}{\PYZhy{}}\PY{l+m+mi}{5}\PY{p}{,}\PY{l+m+mi}{10}\PY{o}{+}\PY{n}{steps}\PY{p}{,} \PY{n}{steps}\PY{p}{)}
        \PY{n}{y} \PY{o}{=} \PY{n}{signalfunc}\PY{p}{(}\PY{n}{w}\PY{p}{)}       
        
        \PY{n}{plt}\PY{o}{.}\PY{n}{title}\PY{p}{(}\PY{l+s+s1}{\PYZsq{}}\PY{l+s+s1}{Derived Function}\PY{l+s+s1}{\PYZsq{}}\PY{p}{)}
        \PY{n}{plt}\PY{o}{.}\PY{n}{plot}\PY{p}{(}\PY{n}{w}\PY{p}{,}\PY{n}{y}\PY{p}{)}
        \PY{n}{plt}\PY{o}{.}\PY{n}{grid}\PY{p}{(}\PY{k+kc}{True}\PY{p}{)}
        \PY{n}{plt}\PY{o}{.}\PY{n}{ylabel}\PY{p}{(}\PY{l+s+s1}{\PYZsq{}}\PY{l+s+s1}{y(t)}\PY{l+s+s1}{\PYZsq{}}\PY{p}{)}
        \PY{n}{plt}\PY{o}{.}\PY{n}{xlabel}\PY{p}{(}\PY{l+s+s1}{\PYZsq{}}\PY{l+s+s1}{t}\PY{l+s+s1}{\PYZsq{}}\PY{p}{)}
        \PY{n}{plt}\PY{o}{.}\PY{n}{show}\PY{p}{(}\PY{p}{)}
\end{Verbatim}

    \begin{center}
    \adjustimage{max size={0.9\linewidth}{0.9\paperheight}}{output_16_0.png}
    \end{center}
    { \hspace*{\fill} \\}
    
    \hypertarget{part-3}{%
\subsection{Part 3:}\label{part-3}}

The final sections' deliverables summed up to showing that the functions
worked properly by manipulating the derived function by shifting time.
The first tasks plot shows a time reversal of the the function,
essentially flipping it about the y-axis (at 0.0) and plotting the
reversed function.

    \begin{Verbatim}[commandchars=\\\{\}]
{\color{incolor}In [{\color{incolor}8}]:} \PY{n}{steps} \PY{o}{=} \PY{l+m+mf}{1e\PYZhy{}3}
        \PY{n}{w} \PY{o}{=} \PY{n}{np}\PY{o}{.}\PY{n}{arange}\PY{p}{(}\PY{o}{\PYZhy{}}\PY{l+m+mi}{10}\PY{p}{,}\PY{l+m+mi}{5}\PY{o}{+}\PY{n}{steps}\PY{p}{,} \PY{n}{steps}\PY{p}{)}
        \PY{n}{y} \PY{o}{=} \PY{n}{signalfunc}\PY{p}{(}\PY{o}{\PYZhy{}}\PY{n}{w}\PY{p}{)}      
        
        \PY{n}{plt}\PY{o}{.}\PY{n}{title}\PY{p}{(}\PY{l+s+s1}{\PYZsq{}}\PY{l+s+s1}{Derived Function With Time Reverse}\PY{l+s+s1}{\PYZsq{}}\PY{p}{)}
        \PY{n}{plt}\PY{o}{.}\PY{n}{plot}\PY{p}{(}\PY{n}{w}\PY{p}{,}\PY{n}{y}\PY{p}{)}
        \PY{n}{plt}\PY{o}{.}\PY{n}{grid}\PY{p}{(}\PY{k+kc}{True}\PY{p}{)}
        \PY{n}{plt}\PY{o}{.}\PY{n}{ylabel}\PY{p}{(}\PY{l+s+s1}{\PYZsq{}}\PY{l+s+s1}{y(t)}\PY{l+s+s1}{\PYZsq{}}\PY{p}{)}
        \PY{n}{plt}\PY{o}{.}\PY{n}{xlabel}\PY{p}{(}\PY{l+s+s1}{\PYZsq{}}\PY{l+s+s1}{t}\PY{l+s+s1}{\PYZsq{}}\PY{p}{)}
        \PY{n}{plt}\PY{o}{.}\PY{n}{show}\PY{p}{(}\PY{p}{)}
\end{Verbatim}

    \begin{center}
    \adjustimage{max size={0.9\linewidth}{0.9\paperheight}}{output_18_0.png}
    \end{center}
    { \hspace*{\fill} \\}
    
    The second task was to time-shift the function by -4 and +4 and plot
both functions on the same plot.

    \begin{Verbatim}[commandchars=\\\{\}]
{\color{incolor}In [{\color{incolor}9}]:} \PY{n}{steps} \PY{o}{=} \PY{l+m+mf}{1e\PYZhy{}3}
        \PY{n}{w} \PY{o}{=} \PY{n}{np}\PY{o}{.}\PY{n}{arange}\PY{p}{(}\PY{o}{\PYZhy{}}\PY{l+m+mi}{20}\PY{p}{,}\PY{l+m+mi}{20}\PY{o}{+}\PY{n}{steps}\PY{p}{,} \PY{n}{steps}\PY{p}{)}
        \PY{n}{y} \PY{o}{=} \PY{n}{signalfunc}\PY{p}{(}\PY{n}{w}\PY{o}{\PYZhy{}}\PY{l+m+mi}{4}\PY{p}{)}
        \PY{n}{x} \PY{o}{=} \PY{n}{signalfunc}\PY{p}{(}\PY{o}{\PYZhy{}}\PY{n}{w}\PY{o}{\PYZhy{}}\PY{l+m+mi}{4}\PY{p}{)}
        
        \PY{n}{plt}\PY{o}{.}\PY{n}{title}\PY{p}{(}\PY{l+s+s1}{\PYZsq{}}\PY{l+s+s1}{Altered Derived Functions }\PY{l+s+s1}{\PYZsq{}}\PY{p}{)}
        \PY{n}{plt}\PY{o}{.}\PY{n}{plot}\PY{p}{(}\PY{n}{w}\PY{p}{,}\PY{n}{y}\PY{p}{,}\PY{n}{label}\PY{o}{=}\PY{l+s+s1}{\PYZsq{}}\PY{l+s+s1}{Original}\PY{l+s+s1}{\PYZsq{}}\PY{p}{)}
        \PY{n}{plt}\PY{o}{.}\PY{n}{plot}\PY{p}{(}\PY{n}{w}\PY{p}{,}\PY{n}{x}\PY{p}{,}\PY{n}{label}\PY{o}{=}\PY{l+s+s1}{\PYZsq{}}\PY{l+s+s1}{Time Shifted}\PY{l+s+s1}{\PYZsq{}}\PY{p}{)}
        \PY{n}{plt}\PY{o}{.}\PY{n}{grid}\PY{p}{(}\PY{k+kc}{True}\PY{p}{)}
        \PY{n}{plt}\PY{o}{.}\PY{n}{ylabel}\PY{p}{(}\PY{l+s+s1}{\PYZsq{}}\PY{l+s+s1}{y(t)}\PY{l+s+s1}{\PYZsq{}}\PY{p}{)}
        \PY{n}{plt}\PY{o}{.}\PY{n}{xlabel}\PY{p}{(}\PY{l+s+s1}{\PYZsq{}}\PY{l+s+s1}{t}\PY{l+s+s1}{\PYZsq{}}\PY{p}{)}
        \PY{n}{plt}\PY{o}{.}\PY{n}{legend}\PY{p}{(}\PY{p}{)}
        \PY{n}{plt}\PY{o}{.}\PY{n}{show}\PY{p}{(}\PY{p}{)}
\end{Verbatim}

    \begin{center}
    \adjustimage{max size={0.9\linewidth}{0.9\paperheight}}{output_20_0.png}
    \end{center}
    { \hspace*{\fill} \\}
    
    The third task was to time-scale the function by lengthening time by two
as well as shortening time by half and graphing both functions on the
same plot.

    \begin{Verbatim}[commandchars=\\\{\}]
{\color{incolor}In [{\color{incolor}10}]:} \PY{n}{steps} \PY{o}{=} \PY{l+m+mf}{1e\PYZhy{}3}
         \PY{n}{w} \PY{o}{=} \PY{n}{np}\PY{o}{.}\PY{n}{arange}\PY{p}{(}\PY{o}{\PYZhy{}}\PY{l+m+mi}{1}\PY{p}{,}\PY{l+m+mi}{20}\PY{o}{+}\PY{n}{steps}\PY{p}{,} \PY{n}{steps}\PY{p}{)}
         \PY{n}{y} \PY{o}{=} \PY{n}{signalfunc}\PY{p}{(}\PY{n}{w}\PY{o}{/}\PY{l+m+mi}{2}\PY{p}{,}\PY{p}{)}
         \PY{n}{x} \PY{o}{=} \PY{n}{signalfunc}\PY{p}{(}\PY{n}{w}\PY{o}{*}\PY{l+m+mi}{2}\PY{p}{,}\PY{p}{)}
         
         \PY{n}{plt}\PY{o}{.}\PY{n}{title}\PY{p}{(}\PY{l+s+s1}{\PYZsq{}}\PY{l+s+s1}{Altered Derived Functions}\PY{l+s+s1}{\PYZsq{}}\PY{p}{)}
         \PY{n}{plt}\PY{o}{.}\PY{n}{plot}\PY{p}{(}\PY{n}{w}\PY{p}{,}\PY{n}{y}\PY{p}{,}\PY{n}{label}\PY{o}{=}\PY{l+s+s1}{\PYZsq{}}\PY{l+s+s1}{Lengthened}\PY{l+s+s1}{\PYZsq{}}\PY{p}{)}
         \PY{n}{plt}\PY{o}{.}\PY{n}{plot}\PY{p}{(}\PY{n}{w}\PY{p}{,}\PY{n}{x}\PY{p}{,}\PY{n}{label}\PY{o}{=}\PY{l+s+s1}{\PYZsq{}}\PY{l+s+s1}{Shortened}\PY{l+s+s1}{\PYZsq{}}\PY{p}{)}
         \PY{n}{plt}\PY{o}{.}\PY{n}{grid}\PY{p}{(}\PY{k+kc}{True}\PY{p}{)}
         \PY{n}{plt}\PY{o}{.}\PY{n}{ylim}\PY{p}{(}\PY{p}{[}\PY{o}{\PYZhy{}}\PY{l+m+mi}{5}\PY{p}{,}\PY{l+m+mi}{10}\PY{p}{]}\PY{p}{)}
         \PY{n}{plt}\PY{o}{.}\PY{n}{xlim}\PY{p}{(}\PY{p}{[}\PY{o}{\PYZhy{}}\PY{l+m+mi}{5}\PY{p}{,}\PY{l+m+mi}{20}\PY{p}{]}\PY{p}{)}
         \PY{n}{plt}\PY{o}{.}\PY{n}{legend}\PY{p}{(}\PY{p}{)}
         \PY{n}{plt}\PY{o}{.}\PY{n}{ylabel}\PY{p}{(}\PY{l+s+s1}{\PYZsq{}}\PY{l+s+s1}{y(t)}\PY{l+s+s1}{\PYZsq{}}\PY{p}{)}
         \PY{n}{plt}\PY{o}{.}\PY{n}{xlabel}\PY{p}{(}\PY{l+s+s1}{\PYZsq{}}\PY{l+s+s1}{t}\PY{l+s+s1}{\PYZsq{}}\PY{p}{)}
         \PY{n}{plt}\PY{o}{.}\PY{n}{show}\PY{p}{(}\PY{p}{)}
\end{Verbatim}

    \begin{center}
    \adjustimage{max size={0.9\linewidth}{0.9\paperheight}}{output_22_0.png}
    \end{center}
    { \hspace*{\fill} \\}
    
    The fourth task was the draw the derivative of the function \$ f(t) \$
and upload the image for comparison with the plot the computer creates
in the last task (task 5). The hand drawn plot is seen below.

    \begin{figure}
\centering
\includegraphics{attachment:Hand_drawn_graph.jpg}
\caption{Hand\_drawn\_graph.jpg}
\end{figure}

    The last task was to plot the derivative of the function and graphing
the derivative along with the original on the same plot. This was also
the part where the TA provided the code to produce the derivative plot
of the original function.

    \begin{Verbatim}[commandchars=\\\{\}]
{\color{incolor}In [{\color{incolor}11}]:} \PY{n}{y} \PY{o}{=} \PY{n}{signalfunc}\PY{p}{(}\PY{n}{w}\PY{p}{)}
         \PY{n}{dt} \PY{o}{=} \PY{n}{np}\PY{o}{.}\PY{n}{diff}\PY{p}{(}\PY{n}{w}\PY{p}{)}
         \PY{n}{dy} \PY{o}{=} \PY{n}{np}\PY{o}{.}\PY{n}{diff}\PY{p}{(}\PY{n}{y}\PY{p}{,} \PY{n}{axis}\PY{o}{=}\PY{l+m+mi}{0}\PY{p}{)}\PY{o}{/}\PY{n}{dt}
         
         \PY{n}{plt}\PY{o}{.}\PY{n}{plot}\PY{p}{(}\PY{n}{w}\PY{p}{,}\PY{n}{y}\PY{p}{,}\PY{n}{label}\PY{o}{=}\PY{l+s+s1}{\PYZsq{}}\PY{l+s+s1}{Original Function}\PY{l+s+s1}{\PYZsq{}}\PY{p}{)}
         \PY{n}{plt}\PY{o}{.}\PY{n}{plot}\PY{p}{(}\PY{n}{w}\PY{p}{[}\PY{n+nb}{range}\PY{p}{(}\PY{n+nb}{len}\PY{p}{(}\PY{n}{dy}\PY{p}{)}\PY{p}{)}\PY{p}{]}\PY{p}{,}\PY{n}{dy}\PY{p}{[}\PY{p}{:}\PY{p}{,}\PY{l+m+mi}{0}\PY{p}{]}\PY{p}{,}\PY{n}{label}\PY{o}{=}\PY{l+s+s1}{\PYZsq{}}\PY{l+s+s1}{Differentiated}\PY{l+s+s1}{\PYZsq{}}\PY{p}{)}
         \PY{n}{plt}\PY{o}{.}\PY{n}{ylim}\PY{p}{(}\PY{p}{[}\PY{o}{\PYZhy{}}\PY{l+m+mi}{2}\PY{p}{,}\PY{l+m+mi}{10}\PY{p}{]}\PY{p}{)}
         \PY{n}{plt}\PY{o}{.}\PY{n}{grid}\PY{p}{(}\PY{k+kc}{True}\PY{p}{)}
         \PY{n}{plt}\PY{o}{.}\PY{n}{title}\PY{p}{(}\PY{l+s+s1}{\PYZsq{}}\PY{l+s+s1}{Differentiated Function Comparison}\PY{l+s+s1}{\PYZsq{}}\PY{p}{)}
         \PY{n}{plt}\PY{o}{.}\PY{n}{legend}\PY{p}{(}\PY{p}{)}
         \PY{n}{plt}\PY{o}{.}\PY{n}{ylabel}\PY{p}{(}\PY{l+s+s1}{\PYZsq{}}\PY{l+s+s1}{y`(t)}\PY{l+s+s1}{\PYZsq{}}\PY{p}{)}
         \PY{n}{plt}\PY{o}{.}\PY{n}{xlabel}\PY{p}{(}\PY{l+s+s1}{\PYZsq{}}\PY{l+s+s1}{t}\PY{l+s+s1}{\PYZsq{}}\PY{p}{)}
         \PY{n}{plt}\PY{o}{.}\PY{n}{show}\PY{p}{(}\PY{p}{)}
\end{Verbatim}

    \begin{center}
    \adjustimage{max size={0.9\linewidth}{0.9\paperheight}}{output_26_0.png}
    \end{center}
    { \hspace*{\fill} \\}
    
    \hypertarget{error-analysis}{%
\section{Error Analysis:}\label{error-analysis}}

Difficulties I encountered for this lab summed up to a misunderstanding
of exactly how to construct the code for the production of the derived
function plot. I was attempting to create one large function that would
take in inputs as arguments, those inputs were the slope and time shift
of the derived function, and then produce the plot for the derived
function. While this would have worked, it would have created a lot more
work as there was no modularity or a way to easily time-shift or
time-scale the final results. To correct this I reached out the the
labs' TA who met with me, guided me onto the right track and thought
process, and then walked me through the rest of the lab. This ensured I
properly understood the reason for this lab as a prerequisit for the
rest of the semester so I wouldn't end up behind later by over
complicating this lab.

    \hypertarget{questionsconclusions}{%
\section{Questions/Conclusions:}\label{questionsconclusions}}

\hypertarget{q1-are-the-plots-from-part-3-tasks-4-and-5-identical-is-it-possible-for-them-to-match-explain-why-or-why-not.}{%
\paragraph{Q1: Are the plots from Part 3 Tasks 4 and 5 identical? Is it
possible for them to match? Explain why or why
not.}\label{q1-are-the-plots-from-part-3-tasks-4-and-5-identical-is-it-possible-for-them-to-match-explain-why-or-why-not.}}

A: The plots from task 4 and 5 are not identical but they are fairly
close. I believe it is possible for them to match but I think the
difference lies somewhere in how the computer handles the math compared
to how human beings handle the math.

\hypertarget{q2-how-does-the-correlation-between-the-two-plots-from-part-3-tasks-4-and-5-change-if-you-were-to}{%
\paragraph{Q2: How does the correlation between the two plots (from Part
3 Tasks 4 and 5) change if you were
to}\label{q2-how-does-the-correlation-between-the-two-plots-from-part-3-tasks-4-and-5-change-if-you-were-to}}

\hypertarget{change-the-step-size-within-the-time-variable-in-task-5-explain-why-this-happens.}{%
\paragraph{change the step size within the time variable in Task 5?
Explain why this
happens.}\label{change-the-step-size-within-the-time-variable-in-task-5-explain-why-this-happens.}}

A: Perhaps I am misunderstanding, maybe not. I added a ``2\emph{" to the
"w" in the function call for dt=np.diff(w) and observed the change. When
the "2}'' was added the graph shifted down the y-axis by one. I
increased the value up to 10 and the graph continued to shift further
down but the step value of 1 between x=(0,5) never moved all the way
down to touch the x-axis. I imagine it has something to do with the
derivative of the time value of the function but I am uncertain of the
exact reason.

\hypertarget{q3-in-what-way-can-the-expectations-and-tasks-be-more-clearly-explained-for-this-lab}{%
\paragraph{Q3: In what way can the expectations and tasks be more
clearly explained for this
lab?}\label{q3-in-what-way-can-the-expectations-and-tasks-be-more-clearly-explained-for-this-lab}}

A: The expectations were perfectly clear for this lab and the last two
labs. My issue, and perhaps I should have made sure I was understanding
this properly in the first place, was my confusion on exactly how to go
about implementing the functions. Maybe a clearer example other than
just change ``sin'' to ``cos'' would help those in the same boat as
myself. I don't expect the instructors to have known exactly what level
all the students were at as far as python knowledge or coding in
general. The TA's willingness to help in person and ability to make
things clear were immensely helpful though. I am certain that having
access to the lab 12 hours before the actual lab will help alleviate
these issues in future labs.

    https://github.com/cmwiegert/ECE351\_Code.git


    % Add a bibliography block to the postdoc
    
    
    
    \end{document}
